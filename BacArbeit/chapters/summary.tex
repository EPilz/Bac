%%%%%%%%%%%%%%%%%%%%%%%%%%%%%%%%%%%%%%%%%%%%%%%%%%%%%%%%%%%%%%%%%%%%%%%%
\chapter{Zusammenfassung}
\label{sec:summary}
%%%%%%%%%%%%%%%%%%%%%%%%%%%%%%%%%%%%%%%%%%%%%%%%%%%%%%%%%%%%%%%%%%%%%%%%
Diese Arbeit hat sich mit der Evaluierung von REST Frameworks für Android im Kontext des Revex2020 Projektes auseinandergesetzt. Die ausgewählten Frameworks wurden anhand einer Beispielimplementierung vorgestellt und verglichen.
\\\\
Zu Beginn der Arbeit wurde auf einen der größten Trends am Business-Markt eingegangen. Dieser ist die Mobilisierung der Geschäftswelt, wobei unabhängig von Stakeholdern, Zeit, Ort und Geräten auf Daten und Anwendungen zugriffen werden soll. Ein wesentlicher Innovationsstrang ist dabei die Entwicklung von Business-Apps. Mobile Endgräte werden in bestehende Geschäftsprozesse der Unternehmen integriert und greifen dabei oft auf definierte Schnittstellen oder angebotene Dienste zu  \cite{smartMobileApps1}. Herkömmliche Software rückt dabei immer weiter in den Hintergrund, Daten sind sofort und überall abrufbar. Mobile Endgeräte wie Smartphones und Tablets verändern daher die Geschäftswelt nachhaltig, da Unternehmensinformationen jederzeit verfügbar sind - die Unternehmen der Zukunft sind mobil \cite{smartMobileApps7}.
\\\\
In einem Teil der Arbeit werden die verwendeten Technologien Android und REST beschrieben. Android ist ein Betriebssystems, welches primär für Smartphones und Tablets konzeptioniert ist  \cite{overviewAndroid:singh}. Durch die quelloffene Struktur des Betriebssystems ist Android bei vielen Konsumenten und Entwicklern sehr beliebt, wodurch viele Unternehmen ihre mobilen Applikationen auf dieses Betriebssystem ausrichten \cite{statsticMobileOS}. Aufgrund dessen wurde die Evaluierung der REST Frameworks auch auf Android ausgerichtet. In einer kurze Zusammenfassung wurde auf REST als Architekturstil eingegangenen und die 5 Kernprinzipien dieses Stils beschrieben \cite{restHttp:book}.
\\\\
Mithilfe der prototypisch entwickelten Android-App soll es Mitarbeitern zukünftig möglich sein, den Zustand einzelner Kraftwerkskomponenten vor Ort zu erfassen oder Details zum Kraftwerk anzuzeigen. Die App soll auf das bereits vorhandene Backend, eines REST-Webservices zugreifen und diesen verwenden. Um auf das vorhandene Backend zuzugreifen muss ein REST Framework verwendet werden, dass für Android kompatibel ist und performant arbeitet. Um geeignete Rest Frameworks zu finden, wurde eine Technologierecherche durchgeführt und die drei populärsten und den Anforderungen adäquatesten Frameworks für eine Beispielimplementierung ausgewählt.
\\\\
Die Qualität der einzelnen Frameworks wurde anhand definierter Kriterien evaluiert. Diese Kriterien umfassen sowohl die Entwicklungskultur, den Implementierungsprozess und einen Performance- und Speichervergleich. Die ausgewählten REST Frameworks wurden anhand einer Beispielimplementierung vorgestellt und deren Verwendungsweise beschrieben. Im Anschluss daran wurden die einzelnen Frameworks mittels der definierten Kriterien verglichen. Durch die Evaluierung der Frameworks konnte festgestellt werden, dass AndroidAnnotations am besten für die Implementierung der Revex2020 App geeignet ist. Als Abschluss wurde noch ein kurzes persönliches Resümee zu den Frameworks gezogen.